\documentclass{article}
\pagestyle{plain}
\addtolength{\textwidth}{20pt}

\newenvironment{chunk}[1]
{\par\smallskip\noindent\textbf{#1.}\par\nopagebreak}
{\par\smallskip}
\newcounter{exnum}
\newenvironment{example}[1]
{\par\smallskip\noindent\refstepcounter{exnum}\textbf{Example
    \theexnum.} The program #1.\par\nopagebreak}
{\par\smallskip}
\renewcommand{\tt}{\texttt}
\newcommand{\synopsis}{\par\noindent\textbf{Synopsis.}}
\newcommand{\synocont}{\\\phantom{\textbf{Synopsis.}}}
\newcommand{\descr}{\par\noindent\textbf{Description.}\ }
\newcommand{\return}{\par\noindent\textbf{Returns:}\ }
\newcommand{\bsl}{\symbol{92}}
\begin{document}
\title{The \tt{optparse.c} module}
\author{R.A. Litherland}
\date{}
\maketitle

\section{Introduction}
This module, consisting of the files \tt{optparse.c} and
\tt{optparse.h},  provides functions to parse the optional
arguments to a program, and to print a help message listing all
options. It was inspired the Python module optparse, which is
referred to below as \tt{optparse.py}, and attempts to make it easy to
get most of the functionality provided by \tt{optparse.py} in a C
program. Options can have one 
or more short forms such as \tt{-h}, a long form such as
\tt{--help}, or both, but only one long form is allowed. Short
forms can be combined in a single argument, and long
forms can be abbreviated if the result is unambiguous (and at least
two characters long). An option may
require futher information (e.g., a filename), which I call the
\emph{value} of the option. (Options requiring more than one value can
be implemented using the function \tt{opt\_remainder}; outside the
description of that function, this possibility is ignored.) 
When an option given in short form needs a value, the
value is the rest of the option argument if that has at least one
character, and the next argument if not. The value of an option given
in long form can be given in the same argument, as in
\tt{--file=foo.txt}, or as the next argument. 
\begin{chunk}{Terminology}
  The words ``argument'', ``help'' and ``usage'' will, I'm afraid, be
  somewhat over-used. An \emph{argument} is either what you pass to a
  C function, or one of a list of strings you pass to the
  \tt{opt\_parse} function. The list is typically the \tt{argv} of
  your \tt{main} function, so I will sometimes refer to these
  strings as \emph{program arguments}. However, as mentioned above, I
  refer to the `value' rather than the `argument' of an
  option. Program arguments are divided into three classes:
  \emph{option arguments}, which specify one or more options;
  \emph{value arguments}, which supply the values of options, and
  \emph{positional arguments}, which are all the rest.

  The \emph{program help message} is everything that gets printed when
  the user gives the help option to your program. Figure
  \ref{fig:fudd-help} shows an example, taken from the documentation
  of \tt{optparse.py} (see pp. 279--280 of the Python Library Reference).
  \begin{figure}\small
\begin{verbatim}
leibniz:~/optparse> ./fudd -h
usage: fudd [options] arg1 arg2
options:
  -h, --help           print this help message and exit
  -v, --verbose        make lots of noise [default]
  -q, --quiet          be vewwy quiet (I'm hunting wabbits)
  -fFILE, --file=FILE  write output to FILE
  -mMODE, --mode=MODE  interaction mode: one of 'novice', 'intermediate'
                       [default], 'expert'
  Dangerous Options:
    Caution: use of these options is at your own risk.  It is believed that
    some of them bite.
    -g                 Group option.
\end{verbatim}
    \caption{A program help message.}\label{fig:fudd-help}
  \end{figure}
The first line, \\
\verb+usage: fudd [options] arg1 arg2+\\
is the \emph{program usage string}. An item like \\
\verb+  -h, --help           print this help message and exit+\\
is an \emph{option usage string}, which is made up of the \emph{option
  names} (\tt{-h, --help}) and the \emph{help text}. Everything else
is \emph{filler}.

  Finally, by an \emph{empty string} I mean a value \tt{s} of type
  \tt{char*} satisfying \tt{!s || !*s}. 
\end{chunk}

In the following sections, I describe all functions, types and macros
declared in \tt{optparse.h}; their names all begin with \tt{opt\_} or
\tt{OPT\_}. 
 
\section{Specifying options}
\begin{chunk}{The type struct opt\_spec}
  \synopsis \verb+ struct opt_spec {+
  \synocont \verb+     int (*action)(char *, void *);+
  \synocont \verb+     const char *sf, *lf, *metavar, *help;+
  \synocont \verb+     void *data;+
  \synocont \verb+ };+
  \descr This struct is used to specify
  options to the \tt{opt\_parse} function. Suppose that \tt{o} has
  type \tt{struct opt\_spec}. A null pointer in \tt{o.action} marks
  the end of an array, and the special value \tt{opt\_text} signals
  that this item does not specify an option, but just provides text to
  be included in the program help message; this case is described
  later. Otherwise, \tt{o.action} is called when the option specified
  is encountered, with its second argument set to \tt{o.data}. For an
  action you define, \tt{o.data} can be anything you
  like\footnote{Probably \tt{NULL}.}. Several pre-defined functions
  that can be used as the \tt{o.action} field are described in
  \S\ref{sect:predef}, and what they expect as their second argument
  is specified there. For an option not taking a value, the first
  argument will be \tt{NULL}; for one taking a value, it will point to
  the value. Note that storing this pointer for later use is
  problematical; see the description of \tt{opt\_store\_str} in
  \S\ref{sect:predef}. A non-zero return value terminates option
  processing, causing all later arguments to be positional (except
  possibly the following argument, if it is the value of the current
  one).


  The characters of \tt{o.sf} are the short-form letters, and
  \tt{o.lf} is the long form; obviously, these should not both be
  empty. Unlike \tt{optparse.py}, \tt{optparse.c} does not require
  that the long form 
  begin with two hyphens (or even one). After all, I compiled the
  example programs of \S\ref{sect:examples} using \tt{gcc -ansi
    -pedantic}. Note, however, that if a user misspells a long form
  starting with a single hyphen, a confusing error message will
  probably result since the argument will end up being parsed as a
  string of short-form options. For an option not taking a value,
  \tt{o.metavar} should be empty; for one taking a value, it should be
  non-empty, and is used in forming the usage string for the
  option. For instance, if \tt{o.sf} is \tt{"f"}, \tt{o.lf} is
  \tt{"--file"} and \tt{o.metavar} is \tt{"FILE"}, the option names will
  be \tt{-fFILE, --file=FILE}. The field \tt{o.help} is the help
  text. If it is null, the option does not appear in the program help
  message; if it is \tt{""} the option appears with empty help text.
  Any initial spaces are not printed with the help text, but at the
  start of the usage string, increasing its indentation. 

  To include text in the program help message that is not associated with an
  option, set \tt{o.action} to \tt{opt\_text}. Then \tt{o.sf} should
  be empty, and \tt{o.data} has a special meaning: a non-null value
  specifies that, for the purposes of computing the starting position
  for option help text, the option list should be split into two just
  before this item. (See Example \ref{ex:notpython} in
  \S\ref{sect:examples}.) If \tt{o.lf} is empty, \tt{o.help} is filler;
  it is printed with indentation determined by any leading spaces. A
  non-empty value of \tt{o.lf} is used to produce a usage string for a
  positional argument, which is formatted exactly as for an option (so
  \tt{o.metavar} should be empty).
\end{chunk}
\begin{chunk}{Macros}
  The macros
\begin{verbatim}
        OPT_NO_ACTION    OPT_NO_SF     OPT_NO_LF
        OPT_NO_METAVAR   OPT_NO_HELP   OPT_NO_DATA
\end{verbatim}
  all expand to \tt{(void *)0}. They may help you remember which
  fields of an \tt{opt\_spec} struct are being initialised to a
  null pointer when you read your code a week later. 
\end{chunk}

\section{Basic functions}
\begin{chunk}{The function opt\_basename}
  \synopsis\verb+ void opt_basename(char *filename, char separator);+
  \descr Modify \tt{filename} in place to its base name, using
  \tt{separator} as the path separator if it is non-zero, or
  \tt{'/'} otherwise.
\end{chunk}
\begin{chunk}{The function opt\_config}
  \synopsis\verb+ void opt_config(int width, int max_help_position,+
  \synocont\verb+                 int indent, const char *separator);+
  \descr Set parameters controlling the layout of the program help
  message. These all have default values; to use the defaults, given
  in brackets, pass a negative or zero value for \tt{width} [79] or
  \tt{max\_help\_position} [24], negative for \tt{indent} [2], or an
  empty string for \tt{separator} [\verb*+"  "+]. The argument
  \tt{width} gives the width in characters to be used for the help
  message, and \tt{indent} gives the indentation for the option usage
  strings. Between the names of the option and its help text,
  \tt{separator} is printed, padded on the left with spaces, if
  necessary, so that all help text starts in the same column. If
  starting the help text on the same line as the option names would
  cause the help text to start to the right of
  \tt{max\_help\_position}, the line is broken before the separator. 
\end{chunk}
\begin{chunk}{The function opt\_options1st}
  \synopsis\verb+ void opt_options1st(void);+
  \descr By default, options and positional arguments may appear in
  any order. If you call \tt{opt\_options1st} before \tt{opt\_parse},
  the options must appear first; any argument after the first
  positional argument is treated as positional.\footnote{Greg Ward,
    the author of \tt{optparse.py}, believes that this behaviour is
    generally annoying to users. It is, however, used by Python, and
    seems quite natural for such a program.}
\end{chunk}
\begin{chunk}{The function opt\_parse}
  \synopsis\verb+ int opt_parse(const char *usage, +
  \synocont\verb+               struct opt_spec *opts,+
  \synocont\verb+               char **argv);+
  \descr Parse the list of program arguments \tt{argv}. 
  The first program argument, \tt{argv[0]}, is treated as the program
  name, and the 
  arguments to be parsed are the remaining strings, terminated by a null
  pointer. (In help and error messages, \tt{argv[0]} is used as
  is, so you may wish to call \tt{opt\_basename} on it first.) The
  argument \tt{opts} should be an array of 
  structs specifying the options to be recognized, terminated by one
  with a null pointer as its \tt{action} field. The argument
  \tt{usage} specifies the program usage string. It is
  used as the format string for a call to \tt{fprintf}, with
  \tt{argv[0]} as the single following argument. If it is empty,
  the default \verb+"usage: %s [options]"+ is used, which is fine
  for a program taking no positional arguments. By default, the next
  line of the help message is \tt{options:}, but if the first
  element of \tt{opts} has \tt{opts[0].action} equal to \tt{opt\_text}
  and \tt{opts[0].lf} empty, this is suppressed.

  \noindent\textbf{Side effect.} Each non-positional argument has its first
  character replaced by 0.\footnote{\label{fn:why0}Yes, yes, it would
    be better if on exit the \tt{argv} array contained only positional
    arguments, but the C standard does not guarantee that it is
    possible to do this.} 

  \noindent\textbf{Special program arguments.} The string \tt{"--"} is treated
  as an option argument that terminates option parsing; all subsequent
  arguments are positional. The string \tt{"-"} is a positional
  argument;\footnote{It can reasonably be argued that this isn't
    actually a special case, since it can't match any option.} note
  that it doesn't terminate option parsing unless you've called
  \tt{opt\_options1st}. 
  
  \return \tt{opt\_parse} returns the number of
  positional arguments.
\end{chunk}
\begin{chunk}{The function opt\_remainder}
  \synopsis\verb+ char ***opt_remainder(void);+
  \descr This should only be used within the action function of an
  option that takes (or can take) more than one value. The
  corresponding \tt{metavar} field should be non-empty, so that
  \tt{opt\_parse} treats it as taking one value. Suppose the return
  value of \tt{opt\_remainder} has been assigned to a variable
  \tt{p}. Then \tt{*p} is the part of the \tt{argv} array supplied to
  \tt{opt\_parse} that has not yet been examined. The effect
  of incrementing \tt{*p} $k$ times is to render $k$ arguments
  invisible to \tt{opt\_parse}; note that your function is responsible
  for zeroing out those arguments. Here is the skeleton of an action
  function for an option requiring two values. 
\begin{verbatim}
int gobble(char *val, void *data)
{
    char ***p = opt_remainder();
    if (!**p) opt_err("option %s requires two values");
    /* Do something with the value strings val and **p. */
    return **(*p)++ = 0;
}
\end{verbatim}
  \return \tt{opt\_remainder} returns a pointer to the variable
  used by \tt{opt\_parse} to step through the program arguments.
\end{chunk}
\section{Error-reporting functions}
The following functions should only be called from within the action
function of an option. The typical reason for doing so is that the
user gave an incorrect value for the option (say $-5$ for an option
expecting a positive integer), but one might also call them if the user
selected an option incompatible with an earlier one. 

\begin{chunk}{The function opt\_name}
  \synopsis\verb+ const char *opt_name(void);+
  \return \tt{opt\_name} returns the name of the option, in
  the form used in the argument, e.g. \tt{"-f"} or \tt{"--file"}, for
  use in error messages.
\end{chunk}
\begin{chunk}{The function opt\_err}
  \synopsis\verb+ void opt_err(const char *msg);+
  \descr Print an error message determined by \tt{msg} and exit. The
  body of this function is:
\begin{verbatim}
    opt_err_pfx();
    fprintf(stderr, msg, opt_name());
    opt_err_sfx();
\end{verbatim}
  If you want a message depending on run-time information other than
  \tt{opt\_name()}, you can replace the second line.
\end{chunk}
\begin{chunk} {The function opt\_err\_pfx}
  \synopsis\verb+ void opt_err_pfx(void);+
  \descr Print the prefix to an error message on \tt{stderr};
  currently, this is the program name followed by \verb*+": "+. 
\end{chunk}
\begin{chunk} {The function opt\_err\_sfx}
  \synopsis\verb+ void opt_err_sfx(void);+
  \descr Print the suffix to an error message on \tt{stderr}
  and exit. Currently, this prints \tt{"\bsl noption usage:\bsl n"}
  followed by the usage string of the option in  question.
\end{chunk}
\section{Pre-defined actions}\label{sect:predef}
These are primarily intended to be stored in the \tt{action} field
of an \tt{opt\_spec} struct, but can also be called from within an
action function you define. For each function, any constraints on its
arguments are specified; after the declaration of the function in the
synopsis there are up to two lines of code that could appear in the
body of the function. One is an assertion that the first argument is
null, or that it isn't null, which must be satisfied. The other
assigns the second argument to a pointer to a specified type; the
accesses through this pointer mentioned in the description must be
valid. (Recall that when a function is stored as \tt{o.action}, it
will be called with arguments depending on \tt{o.metavar} and
\tt{o.data}.) 
\begin{chunk}{The function opt\_text}
  \synopsis\verb+ int opt_text(char *val, void *data);+
  \descr This is a special case, whose only purpose is to be stored in
  an \tt{action} field; it should never be called.
\end{chunk}
\begin{chunk}{The function opt\_help}
  \synopsis\verb+ int opt_help(char *val, void *data);+
  \synocont\verb+ assert(val == 0);+
  \descr Print the program help message and exit. If this function is
  stored as \tt{o.action} and \tt{o.help} is empty, \tt{opt\_parse}
  will use the default help text

  \tt{"print this help message and exit"}. 
  \return \tt{opt\_help} never returns.
\end{chunk}
\begin{chunk}{The function opt\_version}
  \synopsis\verb+ int opt_version(char *val, void *data);+
  \synocont\verb+ const char *p = data;+
  \synocont\verb+ assert(val == 0);+
  \descr Print the version string \tt{p} and exit. If this function is
  stored as \tt{o.action} and \tt{o.help} is empty, \tt{opt\_parse}
  will use the default help text

  \tt{"print the version number and exit"}. 
  \return \tt{opt\_version} never returns.
\end{chunk}
\begin{chunk}{The function opt\_stop}
  \synopsis\verb+ int opt_stop(char *val, void *data);+
  \descr This is the only pre-defined action that returns a non-zero
  value, halting option processing. Other than the return value, it
  acts like \tt{opt\_store\_str} if \tt{val} is non-null, and like
  \tt{opt\_store\_1} if \tt{val} is null and \tt{data} is not, and
  the arguments must satisfy the constraints for the appropriate
  function. If both arguments are null, \tt{opt\_stop} does nothing but
  return a value.
  \return \tt{opt\_stop} returns 1.
\end{chunk}
\begin{chunk}{The function opt\_store\_0}
  \synopsis\verb+ int opt_store_0(char *val, void *data);+
  \synocont\verb+ int *p = data;+
  \synocont\verb+ assert(val == 0);+
  \descr Set \tt{*p} to 0.
  \return \tt{opt\_store\_0} returns 0.
\end{chunk}
\begin{chunk}{The function opt\_store\_1}
  \synopsis\verb+ int opt_store_1(char *val, void *data);+
  \synocont\verb+ int *p = data;+
  \synocont\verb+ assert(val == 0);+
  \descr Set \tt{*p} to 1.
  \return \tt{opt\_store\_1} returns 0.
\end{chunk}
\begin{chunk}{The function opt\_incr}
  \synopsis\verb+ int opt_incr(char *val, void *data);+
  \synocont\verb+ int *p = data;+
  \synocont\verb+ assert(val == 0);+
  \descr Increment \tt{*p}.
  \return \tt{opt\_incr} returns 0.
\end{chunk}
\begin{chunk}{The function opt\_store\_char}
  \synopsis\verb+ int opt_store_char(char *val, void *data);+
  \synocont\verb+ char *p = data;+
  \synocont\verb+ assert(val != 0);+
  \descr If the string \tt{val} has length 1, set \tt{*p} to
  \tt{*val}. Otherwise it is an error.
  \return \tt{opt\_store\_char} returns 0.
\end{chunk}
\begin{chunk}{The function opt\_store\_int}
  \synopsis\verb+ int opt_store_int(char *val, void *data);+
  \synocont\verb+ int *p = data;+
  \synocont\verb+ assert(val != 0);+
  \descr If the string \tt{val} is the decimal representation of an
  integer, set \tt{*p} to that integer. Otherwise it is an error.
  \return \tt{opt\_store\_int} returns 0.
\end{chunk}
\begin{chunk}{The function opt\_store\_int\_lim}
  \synopsis\verb+ int opt_store_int_lim(char *val, void *data);+
  \synocont\verb+ int *p = data;+
  \synocont\verb+ assert(val != 0);+
  \descr The same as \tt{opt\_store\_int}, except that it is an error
  if the integer is less than \tt{p[1]} or greater than \tt{p[2]}.
  \return \tt{opt\_store\_int\_lim} returns 0.
\end{chunk}
\begin{chunk}{The function opt\_store\_double}
  \synopsis\verb+ int opt_store_double(char *val, void *data);+
  \synocont\verb+ double *p = data;+
  \synocont\verb+ assert(val != 0);+
  \descr If the string \tt{val} represents a number, as specified
  for the standard C library function \tt{strtod}, set \tt{*p} to that
  number. Otherwise it is an error. 
  \return \tt{opt\_store\_double} returns 0.
\end{chunk}
\begin{chunk}{The function opt\_store\_double\_lim}
  \synopsis\verb+ int opt_store_double_lim(char *val, void *data);+
  \synocont\verb+ double *p = data;+
  \synocont\verb+ assert(val != 0);+
  \descr The same as \tt{opt\_store\_double}, except that it is an error
  if the number is less than \tt{p[1]} or greater than \tt{p[2]}.
  \return \tt{opt\_store\_double\_lim} returns 0.
\end{chunk}
\begin{chunk}{The type struct opt\_str and the function opt\_store\_str}
  \synopsis\verb+ struct opt_str {char *s, s0;};+
  \synocont\verb+ int opt_store_str(char *val, void *data);+
  \synocont\verb+ struct opt_str *p = data;+
  \synocont\verb+ assert(val != 0);+
  \descr Set \tt{p->s} to \tt{val} and \tt{p->s0} to
  \tt{val[0]}. The reason for doing this is that, if \tt{val} was
  a separate argument, after this function returns \tt{opt\_parse} will
  set \tt{val[0]} to 0. (The reason for \emph{that} is explained in
  footnote \ref{fn:why0} on page \pageref{fn:why0}.) The examples in
  \S\ref{sect:examples} show how the calling program can cope sensibly
  with this situation.
  \return \tt{opt\_store\_str} returns 0.
\end{chunk}
\begin{chunk}{The function opt\_store\_choice}
  \synopsis\verb+ int opt_store_choice(char *val, void *data);+
  \synocont\verb+ const char **p = data;+
  \synocont\verb+ assert(val != 0);+
  \descr Allows the user to choose one of the strings in the
  null-terminated array \tt{p}. If the string \tt{val} compares
  equal (using \tt{strcmp}) to \tt{p[i]} for some \tt{i}, \tt{p[0]}
  and \tt{p[i]} are swapped. Otherwise it is an error, unless some
  \tt{p[i]} is \tt{""}, in which case this is swapped with
  \tt{p[0]}. (In other words, if some \tt{p[i]} is \tt{""}, invalid
  choices are quietly ignored. Probably you'd write an action function
  that calls \tt{opt\_store\_choice} and prints a warning if \tt{""}
  has been ``chosen''.)
  \return \tt{opt\_store\_choice} returns 0.
\end{chunk}
\begin{chunk}{The function opt\_store\_choice\_abbr}
  \synopsis\verb+ int opt_store_choice_abbr(char *val, void *data);+
  \synocont\verb+ const char **p = data;+
  \synocont\verb+ assert(val != 0);+
  \descr The same as \tt{opt\_store\_choice}, except that
  unambiguous abbreviations are accepted, and an ambiguous
  abbreviation is an unconditional error.
  \return \tt{opt\_store\_choice\_abbr} returns 0.
\end{chunk}

\section{Examples}\label{sect:examples}
The C optparse package contains two complete, if useless, programs, in
the files \tt{fudd.c} and 
\tt{notpython.c}. Each program calls \tt{opt\_parse}, and if this
returns prints which options were used to what effect, and any
positional arguments remaining. Below I comment on some aspects of
these programs.
\begin{example}{Fudd}\label{ex:fudd}
  This is based on the example from the \tt{optparse.py} documentation
  referred to earlier, and its help message is shown in Figure
  \ref{fig:fudd-help}. Figures \ref{fig:fudd-top} and
  \ref{fig:fudd-main} show the whole program, with the first part
  setting up the options, and the second being the \tt{main}
  function. 
  \begin{figure}\small
\begin{verbatim}
#include <stdio.h>
#include <stdlib.h>
#include "optparse.h"

int verbose = 1, group = 0;
struct opt_str fn = {NULL, 0};
const char *mode[] = {
    "intermediate", "novice", "expert", NULL
};

struct opt_spec options[] = {
    {opt_help, "h", "--help", OPT_NO_METAVAR, OPT_NO_HELP, OPT_NO_DATA},
    {opt_store_1, "v", "--verbose", OPT_NO_METAVAR,
     "make lots of noise [default]", &verbose},
    {opt_store_0, "q", "--quiet", OPT_NO_METAVAR,
     "be vewwy quiet (I'm hunting wabbits)", &verbose},
    {opt_store_str, "f", "--file", "FILE", "write output to FILE", &fn},
    {opt_store_choice, "m", "--mode", "MODE", "interaction mode: one of "
     "'novice', 'intermediate' [default], 'expert'", mode},
    {opt_text, OPT_NO_SF, OPT_NO_LF, OPT_NO_METAVAR,
     "  Dangerous Options:", OPT_NO_DATA},
    {opt_text, OPT_NO_SF, OPT_NO_LF, OPT_NO_METAVAR,
     "    Caution: use of these options is at your own risk.  "
     "It is believed that some of them bite.", OPT_NO_DATA},
    {opt_store_1, "g", OPT_NO_LF, OPT_NO_METAVAR, "  Group option.", &group},
    {OPT_NO_ACTION}
};
\end{verbatim}
    \caption{Setting up the options for Fudd.}
    \label{fig:fudd-top}
  \end{figure}
  \begin{figure}\small
\begin{verbatim}
int main(int argc, char **argv)
{
   int i;

    opt_basename(argv[0], '/');
    if (opt_parse("usage: %s [options] arg1 arg2", options, argv) != 2) {
        fprintf(stderr, "%s: 2 arguments required\n", argv[0]);
        exit(EXIT_FAILURE);
    }
    printf("noise level: %s\n", verbose ? "verbose" : "quiet");
    if (fn.s) {
        fn.s[0] = fn.s0;
        printf("output file: %s\n", fn.s);
    }
    else
        puts("output file: none");
    printf("interaction mode: %s\n", mode[0]);
    printf("group option: %d\n", group);
    puts("positional arguments:");
    for (i = 1; i < argc; ++i) {
        if (*argv[i] && argv[i] != fn.s)
            printf("  %s\n", argv[i]);
    }
    return 0;
}
\end{verbatim}
    \caption{Fudd's \tt{main} function.}
    \label{fig:fudd-main}
  \end{figure}
  Here are a few points to notice.
  \begin{itemize}
  \item I decided that the user can repeat an option, with all but the
    last use being ignored, and can also use both the conflicting
    options \tt{-v} and \tt{-q}, with the last-used winning. Given
    this, we can use only pre-defined actions.
  \item In \tt{options[0]}, we don't supply help text since the
    default is what we want.
  \item In the array \tt{options}, the strings in the \tt{help} fields
    of the 6th, 7th and 8th members have leading spaces that produce
    the same indentation in the help message as the use of an option
    group in \tt{optparse.py}.
  \item After calling \tt{opt\_parse}, we make sure the value of
    \tt{-f}, if supplied, has its first character correct:
{\small
\begin{verbatim}
    if (fn.s) {
        fn.s[0] = fn.s0;
        printf("output file: %s\n", fn.s);
    }
\end{verbatim}}
    Then, in printing the positional arguments, we have to check that
    a non-empty argument isn't the value of \tt{-f}:
{\small
\begin{verbatim}
    for (i = 1; i < argc; ++i) {
        if (*argv[i] && argv[i] != fn.s)
            printf("  %s\n", argv[i]);
    }
\end{verbatim}}
  \end{itemize}
Here is the output when several options other than \tt{-h}
are given.
{\small
\begin{verbatim}
leibniz:~/optparse> ./fudd string1 -qg -f outfile --mode=expert string2
noise level: quiet
output file: outfile
interaction mode: expert
group option: 1
positional arguments:
  string1
  string2
\end{verbatim}}
\noindent This shows an invalid value given to an option.
{\small
\begin{verbatim}
leibniz:~/optparse> ./fudd --mo movice
fudd: invalid choice 'movice' for option --mode
option usage:
  -mMODE, --mode=MODE  interaction mode: one of 'novice', 'intermediate'
                       [default], 'expert'
\end{verbatim}}
\end{example}
\begin{example}{NotPython}\label{ex:notpython}
  The object of the exercise here is to write a program that accepts
  the same options, and produces the same help message, as Python. The
  help message of NotPython is shown in Figure \ref{fig:notpy-help},
  and you can check that it is the same as for Python 2.3.5 (except
  for the program name in the first line, of course). 
  \begin{figure}\small
\begin{verbatim}
julia:~/optparse> ./notpython -h
usage: notpython [option] ... [-c cmd | file | -] [arg] ...
Options and arguments (and corresponding environment variables):
-c cmd : program passed in as string (terminates option list)
-d     : debug output from parser (also PYTHONDEBUG=x)
-E     : ignore environment variables (such as PYTHONPATH)
-h     : print this help message and exit
-i     : inspect interactively after running script, (also PYTHONINSPECT=x)
         and force prompts, even if stdin does not appear to be a terminal
-O     : optimize generated bytecode (a tad; also PYTHONOPTIMIZE=x)
-OO    : remove doc-strings in addition to the -O optimizations
-Q arg : division options: -Qold (default), -Qwarn, -Qwarnall, -Qnew
-S     : don't imply 'import site' on initialization
-t     : issue warnings about inconsistent tab usage (-tt: issue errors)
-u     : unbuffered binary stdout and stderr (also PYTHONUNBUFFERED=x)
         see man page for details on internal buffering relating to '-u'
-v     : verbose (trace import statements) (also PYTHONVERBOSE=x)
-V     : print the Python version number and exit
-W arg : warning control (arg is action:message:category:module:lineno)
-x     : skip first line of source, allowing use of non-Unix forms of #!cmd
file   : program read from script file
-      : program read from stdin (default; interactive mode if a tty)
arg ...: arguments passed to program in sys.argv[1:]
Other environment variables:
PYTHONSTARTUP: file executed on interactive startup (no default)
PYTHONPATH   : ':'-separated list of directories prefixed to the
               default module search path.  The result is sys.path.
PYTHONHOME   : alternate <prefix> directory (or <prefix>:<exec_prefix>).
               The default module search path uses <prefix>/pythonX.X.
PYTHONCASEOK : ignore case in 'import' statements (Windows).
\end{verbatim}
    \caption{The NotPython help message.}
    \label{fig:notpy-help}
  \end{figure}
  Because of the large number of options, the code is too long to
  reproduce here; see \tt{notpython.c}. We shall give those parts that
  seem noteworthy. The formatting of the help message is different
  from our default, and the options must come first (because Python
  has to pass arguments to the script it interprets, and these may
  look like Python options). Therefore just before the call to
  \tt{opt\_parse} are the lines 
{\small
\begin{verbatim}
    opt_config(0, 0, 0, ": ");
    opt_options1st();
\end{verbatim}}
  \noindent The first two arguments to \tt{opt\_config} select the
  default width and maximum help position, which are large enough not
  to cause unwanted line breaks. The line breaks in the long help
  strings don't correspond to word-wrapping at any width, so we put in
  explicit newlines. Before \tt{main}, we define some variables used
  by action functions, one action function \tt{storeW}, and a longish
  array \tt{options} of \tt{opt\_spec} structs. The second line of the
  help message differs from the default, so the first element of
  \tt{options} is 
{\small
\begin{verbatim}
    {opt_text, OPT_NO_SF, OPT_NO_LF, OPT_NO_METAVAR,
     "Options and arguments (and corresponding environment variables):",
     OPT_NO_DATA},
\end{verbatim}}
There are 14 options; \tt{-OO} is not a separate option, but
represents giving the \tt{-O} option twice, so the corresponding
element of \tt{options} pretends that \tt{-OO} is a positional
variable: 
{\small
\begin{verbatim}
    {opt_text, OPT_NO_SF, "-OO", OPT_NO_METAVAR,
     "remove doc-strings in addition to the -O optimizations", OPT_NO_DATA},
\end{verbatim}}
  \noindent Setting up the \tt{-h} and \tt{-V} options is easy:
{\small
\begin{verbatim}
    {opt_help, "h", OPT_NO_LF, OPT_NO_METAVAR, OPT_NO_HELP, OPT_NO_DATA},
\end{verbatim}}
{\small
\begin{verbatim}
    {opt_version, "V", OPT_NO_LF, OPT_NO_METAVAR,
     "print the Python version number and exit", "This is not Python 2.3.5"},
\end{verbatim}}
  \noindent There are seven options that are simple switches (\tt{-d}, \tt{-E},
  \tt{-i}, \tt{-S}, \tt{-u}, \tt{-v} and \tt{-x}), so we define an
  array
{\small
\begin{verbatim}
int flags[7];
\end{verbatim}}
  \noindent and each switch has an entry like
{\small
\begin{verbatim}
    {opt_store_1, "E", OPT_NO_LF, OPT_NO_METAVAR,
     "ignore environment variables (such as PYTHONPATH)", flags + 1},
\end{verbatim}}
  \noindent The \tt{-O} option has different effects depending on
  whether it is given once or twice, so we define a counter variable 
{\small
\begin{verbatim}
int opt_level;
\end{verbatim}}
  \noindent and an \tt{options} element
{\small
\begin{verbatim}
    {opt_incr, "O", OPT_NO_LF, OPT_NO_METAVAR,
     "optimize generated bytecode (a tad; also PYTHONOPTIMIZE=x)",
     &opt_level},
\end{verbatim}}
  \noindent The variable \tt{opt\_level} will end up holding the
  number of times \tt{-O} was given. This being a toy program, we just
  print this number out; in real life, values greater than 2 could be
  treated as an error, or as equivalent to 2. The \tt{-t} option is
  handled exactly like \tt{-O}. The \tt{-Q} option has a value
  which must be one of four strings. They must be entered exactly (I
  checked), so we define an array giving the choices
{\small
\begin{verbatim}
const char *divchoices[] = {"old", "warn", "warnall", "new", NULL};
\end{verbatim}}
  \noindent and an element of \tt{options}
{\small
\begin{verbatim}
    {opt_store_choice, "Q", OPT_NO_LF, " arg",
     "division options: -Qold (default), -Qwarn, -Qwarnall, -Qnew",
     divchoices},
\end{verbatim}}
  \noindent The option \tt{-c} takes a string value and also
  terminates the option list, which is handled by a variable
{\small
\begin{verbatim}
struct opt_str scriptstr;
\end{verbatim}}
  \noindent and an item in \tt{options}
{\small
\begin{verbatim}
    {opt_stop, "c", OPT_NO_LF, " cmd",
     "program passed in as string (terminates option list)", &scriptstr},
\end{verbatim}}
  \noindent The remaining option, \tt{-W}, is fairly complicated, and
  cannot be handled by a pre-defined action. Since this is just an
  example, it does not cope completely with this option. First,
  multiple \tt{-W} options can be given. I set an arbitrary limit,
{\small
\begin{verbatim}
#define MAXW 10
\end{verbatim}}
  \noindent and if this is exceeded print a warning message and ignore
  extra \tt{-W} options. Next, the value of a \tt{-W} option is a
  colon-separated list of up to 5 fields, of which the first is one of
  6 (possibly abbreviated) strings (the action), and invalid values
  are ignored with a warning. In NotPython, I split the value into the
  first field and the tail. Then I call \tt{opt\_store\_choice\_abbr}
  to find out which of the 6 choices was supplied. If none of them
  was, a warning is given and the option ignored. Otherwise, I store
  the index of the chosen first field, along with the tail, which is
  not checked in any way. This is all implemented by: variables
{\small
\begin{verbatim}
int nW, Wactions[MAXW];
char *Wtails[MAXW];
\end{verbatim}}
  \noindent to hold the number of \tt{-W} options used, the indices of
  their first fields, and their tails; an array
{\small
\begin{verbatim}
const char *Wchoices[] = {
    "ignore", "default", "all", "module", "once", "error", "", NULL
};
\end{verbatim}}
  \noindent giving the choices for the first field, together with
  \tt{""} so that \tt{opt\_store\_choice\_abbr} will not treat an
  invalid choice as an error; the action function \tt{storeW}, shown
  in Figure \ref{fig:storeW};
  \begin{figure}\small
\begin{verbatim}
int storeW(char *val, void *data)
{
    const char *first;
    char *tail;
    int i;

    if (nW == MAXW) {
        fprintf(stderr, "-W options after number %d are ignored.\n", MAXW);
        return 0;
    }
    first = Wchoices[0];
    tail = strchr(val, ':');
    if (tail)
        *tail = 0;
    opt_store_choice_abbr(val, Wchoices);
    for (i = 0; Wchoices[i] != first; ++i)
        ;
    Wchoices[i] = Wchoices[0];
    Wchoices[0] = first;
    if (Wchoices[i][0]) {
        Wactions[nW] = i;
        Wtails[nW++] = tail ? tail + 1 : "";
    }
    else 
        fprintf(stderr, "Invalid -W option ignored: invalid action: '%s'\n",
                val);
    return 0;
}
\end{verbatim}
    \caption{The action function \tt{storeW}.}
    \label{fig:storeW}
  \end{figure}
  and the element
{\small
\begin{verbatim}
    {storeW, "W", OPT_NO_LF, " arg",
     "warning control (arg is action:message:category:module:lineno)",
     OPT_NO_DATA},
\end{verbatim}}
  \noindent of \tt{options}. 

  Towards the end of \tt{options} are three elements giving usage
  strings for positional variables, of which the first is
{\small
\begin{verbatim}
    {opt_text, OPT_NO_SF, "file", OPT_NO_METAVAR,
     "program read from script file", OPT_NO_DATA},
\end{verbatim}} 
\noindent  Then comes a filler element:
{\small
\begin{verbatim}
    {opt_text, OPT_NO_SF, OPT_NO_LF, OPT_NO_METAVAR,
     "Other environment variables:", ""},
\end{verbatim}}
  \noindent Note the use of a non-null \tt{data} field to cause the
  help text for the following items to start in a different column
  from that for the previous items. Finally we have four elements
  giving usage strings for environment variables, of which the first is
{\small
\begin{verbatim}
    {opt_text, OPT_NO_SF, "PYTHONSTARTUP", OPT_NO_METAVAR,
     "file executed on interactive startup (no default)", OPT_NO_DATA},
\end{verbatim}}
  \noindent The part of \tt{main} after the call to \tt{opt\_parse} is
  straightforward.
\end{example}
\end{document}
